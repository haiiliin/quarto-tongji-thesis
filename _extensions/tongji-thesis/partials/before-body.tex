%%% 封面部分
\frontmatter

\tongjisetup{
  %******************************
  % 注意:
  %   1. 配置里面不要出现空行
  %   2. 不需要的配置信息可以删除
  %******************************
  %
  %=====
  % 秘级
  %=====
  secretlevel={保密},
  secretyear={2},
  %
  %=========
  % 中文信息
  %=========
  % 题目过长可以换行(推荐手动加入换行符,这样就可以控制换行的地方啦)。
  ctitle={$title$},
  cheadingtitle={$headingtitle$},    %用于页眉的标题,不要换行
  cauthor={$author$},  
  studentnumber={$studentnumber$},
  cmajorfirst={$cmajorfirst$},
  cmajorsecond={$cmajorsecond$},
  cdepartment={$cdepartment$},
  csupervisor={$csupervisor$}, 
  % 如果没有副指导老师或者校外指导老师,把{}中内容留空即可,或者直接注释掉。
  cassosupervisor={$cassosupervisor$}, % 副指导老师
  % 日期自动使用当前时间,若需手动指定,按如下方式修改:
  % cdate={\zhdigits{2018}年\zhnumber{11}月},
  % 没有基金的话就注释掉吧。
  cfunds={$cfunds$},
  %
  %=========
  % 英文信息
  %=========
  etitle={$etitle$}, 
  eauthor={$eauthor$},
  emajorfirst={$emajorfirst$},
  emajorsecond={$emajorsecond$},
  edepartment={$edepartment$},
  % 日期自动使用当前时间,若需手动指定,按如下方式修改:
  % edate={November,\ 2018},
  esupervisor={$esupervisor$},
  eassosupervisor={$eassosupervisor$},
  efunds={$efunds$},    
}

\ctexset{%
  chapter/name={第,章},
  appendixname=附录,
  contentsname={目录},
  listfigurename=插图索引,
  listtablename=表格索引,
  figurename=图,
  tablename=表,
  bibname=参考文献,
  indexname=索引,
}

% 定义中英文摘要和关键字
\begin{cabstract}
$abstract$
\end{cabstract}

\ckeywords{$for(keywords/allbutlast)$$keywords$; $endfor$
  $for(keywords/last)$$keywords$$endfor$
}

\begin{eabstract}
$eabstract$
\end{eabstract}

\ekeywords{$for(ekeywords/allbutlast)$$ekeywords$; $endfor$
  $for(ekeywords/last)$$ekeywords$$endfor$
}

\makecover

% 目录
\tableofcontents

% % 符号对照表
$denotation$

%%% 以下索引按需要选择
% 插图索引
\listoffigures
% 表格索引
\listoftables
% 公式索引
% \listofequations

%%% 正文 
\mainmatter
